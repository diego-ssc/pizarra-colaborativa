\documentclass[a4paper, oneside, final]{scrartcl}

\usepackage{scrlayer-scrpage}
\usepackage{titlesec}
\usepackage{marvosym}
\usepackage{tabularx,colortbl}
\usepackage{ebgaramond}
\usepackage{microtype}
\usepackage{hyperref}

\titleformat{\section}{\large\scshape\raggedright}{}{0em}{}[\titlerule]

\pagestyle{scrheadings}

\addtolength{\voffset}{-0.5in}
\addtolength{\textheight}{3cm}

\newcommand{\gray}{\rowcolor[gray]{.90}}

\renewcommand{\headfont}{\normalfont\rmfamily}

\begin{document}

\begin{center}
  {\fontsize{10}{10}\selectfont\scshape\textls[50]{Especificaciones Técnicas}}\\
  {\fontsize{10}{10}\selectfont\scshape\textls[50]{Pizarra Colaborativa para
      Computólogos}}\\
  \vspace{1cm}
  {\fontsize{9}{9}\selectfont\scshape\textls[50]{Equipo 1}}\\
  \vspace{0.2cm}
  {\fontsize{7}{7}\selectfont\scshape\textls[50]{Diego Sebastián Sánchez Correa}}\\
  {\fontsize{7}{7}\selectfont\scshape\textls[50]{Nombre}}\\
  {\fontsize{7}{7}\selectfont\scshape\textls[50]{Nombre}}\\
  {\fontsize{7}{7}\selectfont\scshape\textls[50]{Nombre}}\\
  {\fontsize{7}{7}\selectfont\scshape\textls[50]{Nombre}}\\
\end{center}
\vspace{2cm}

\begin{flushright}
  \footnotesize{Creado: 15/02/2024}\\
  \footnotesize{Última vez actualizado: 15/02/2024}\\
\end{flushright}

\section{Resumen}

Se pretende satisfacer las necesidades esenciales para el desarrollo de
proyectos relacionados con las ciencias de la computación.

\section{Glosario}

\section{Contexto}

\section{Requerimientos Técnicos}

\begin{enumerate}
\item \textbf{Integración de Diagramas y Codificación}\\
  Provee un ambiente donde los desarrolladores pueden crear diagramas y escribir
  código en el mismo lugar, facilitando, con ello, la visualización y la
  implementación simultánea de ideas.

\item \textbf{Colaboración Concurrente}\\
  Permite a los desarrolladores trabajar en la misma pizarra de una manera
  simultánea, lo que pretende promover la colaboración en tiempo real y una
  comunicación fluida entre equipos de desarrollo.

\item \textbf{Guardado Automático y Control de Versiones}\\
  Garantiza la seguridad de los datos mediante el guardado automático del
  trabajo hecho en la pizarra; además, ofrece funcionalidades de control de
  versiones para rastrear y administrar cambios.

\item \textbf{Pizarra Infinita}\\
  Ofrece una pizarra virtual sin límites, inspirada en la aplicación
  \textit{Miro}, que permite a los desarrolladores pensar creativamente sin
  preocuparse por limitaciones de espacio.

\item \textbf{Soporte de \LaTeX y graficación de funciones}\\
  Integra herramientas que permiten la creación y graficación de fórmulas de
  \LaTeX, facilitando la representación visual de información compleja.

\item \textbf{Funcionalidades Avanzadas de Colaboración}\\
  Facilita la interacción de usuarios a través de funcionalidades como
  comentarios, menciones y notificaciones en tiempo real; promoviendo un
  ambiente de trabajo colaborativo y eficiente.

  \section{Objetivos fuera del alcance}
  \section{Objetivos futuros}
  \section{Suposiciones}

  \section{Diseño}

\end{enumerate}

\end{document}